%%%%%%%%%%%%%%%%%%%%%%%%%%%%%%%%%%%%%%%%%
% Plantilla para reporte en LaTeX
% Por Daniel Martínez
% dag.mtz.97@gmail.com
%
%
% Basado en:
%
% WikiBooks (http://en.wikibooks.org/wiki/LaTeX/Title_Creation)
%
% License:
% CC BY-NC-SA 3.0 (http://creativecommons.org/licenses/by-nc-sa/3.0/)
%%%%%%%%%%%%%%%%%%%%%%%%%%%%%%%%%%%%%%%%%

% Declarar el tipo de documento
\documentclass[11pt,letterpaper]{exam} 

% Paquetes para idioma, carácteres y símbolos
\usepackage[condensed,math]{iwona}
\usepackage[T1]{fontenc}
\usepackage[spanish, mexico]{babel}
\usepackage{amsfonts}
\usepackage{amssymb}

% Tipo de letra 
% \usepackage{tgbonum}
% \usepackage{tgadventor}

% Márgenes, figuras, gráficos, tablas, listas
%\usepackage{geometry}
\usepackage[top=0.6in, bottom=0.6in, left=0.6in, right=0.6in]{geometry}
\usepackage[justification=centering]{caption}
\usepackage{subcaption}
\usepackage{graphicx}
\usepackage{epstopdf}
\usepackage{tabularx}
\usepackage{float}
\usepackage{enumitem}
%\epstopdfsetup{update} % only regenerate pdf files when eps file is newer
% \usepackage{multirow}
% \usepackage{textcomp}

\usepackage{tikz}

%\usepackage{mathtools}
%\usepackage{siunitx}
%\usepackage{mathrsfs}

% Otros

% \usepackage{hyperref}
% \usepackage[os=win]{menukeys}
% \usepackage{biblatex}
% \addbibresource{sample.bib}

% \usepackage{fancyhdr}
% \usepackage{listingsutf8}
% \usepackage{courier}
% \usepackage{color} %red, green, blue, yellow, cyan, magenta, black, white

% Configuraciones 
% \input{arduinoLanguage.tex}

% \pagestyle{fancy}

% \renewcommand\lstlistingname{Bloque de código}
% \renewcommand\lstlistlistingname{Índice de códigos}

\setlength{\headheight}{16pt}

% \rhead{\leftmark}
% \lhead{\rightmark}
% \cfoot{\thepage}
% \renewcommand{\headrulewidth}{0.4pt}
% \renewcommand{\footrulewidth}{0.4pt}

\newcommand{\HRule}{\rule{\linewidth}{0.5mm}} % Comando para líneas horizontales

% \definecolor{mygreen}{RGB}{28,172,0} % color values Red, Green, Blue
% \definecolor{mylilas}{RGB}{170,55,241}

%%%%%%%%%%%%%%%%%%%%%%%%%%%%%%%%%%%%%%%%%
% Inicio del documento
%%%%%%%%%%%%%%%%%%%%%%%%%%%%%%%%%%%%%%%%%

\begin{document}
    \begin{center}
        \textsc{\Large Universidad de Guadalajara}\\
        \textsc{Centro Universitario de Ciencias Exactas e Ingenierías}\\
        \vspace{6mm}
        \textbf{Examen parcial \#1\\ Programación de Sistemas Embebidos - I7266}\\
        \vspace{6mm}
        {\small Nombre: \rule{0.7\textwidth}{0.4pt}} Fecha: \rule{0.15\textwidth}{0.4pt}\\
        \vspace{6mm}
        \fbox{\fbox{\parbox{6in}{\centering You have 20 minutes to answer as many of the following questions as you can. Each question gives you a certain number of points, you will only obtain them if your answer is correct, if your answer is partially correct, you will obtain a proportional part of the points.}}}
    \end{center}
    \begin{questions}
        \question What do you understand by \textit{microcontroller architechture}? (15 points)

        \question Describe the primary function of a microcontroller in an embedded system. (5 points)
    
        \question What is a register? (5 points)
        \begin{parts}
            \part What are registers used for? (10 points)
            
            \part What are the differences between registers and RAM memory? (10 points)
        \end{parts}
    
        
        \question Explain the differences between \textit{volatile} and \textit{non-volatile} memory. (5 points)
        \begin{parts}
            \part Is RAM volatile? What does RAM mean and what is it used for? (10 points)
            
            \part Is ROM volatile? What does ROM mean and what is it used for? (10 points)
        \end{parts}
        
        \question What is an interrupt?  (5 points) 
        \begin{parts}
            \part What is the difference between a software interrupt and a hardware interrupt? (10 points)
            
            \part What is the advantage of using an interrupt against polling for an event? (10 points)
        \end{parts}

        \question In an 16-bit microcontroller, the last reachable address is \texttt{65,535} in decimal. In a 32-bit microcontroller, what would be the last addressable memory location? (15 points)
        
        \clearpage
        \question Why do we need a compiler and how does a compiler work? (20 points)

        \question In the following diagram 
        
    \end{questions}

\end{document}
%\vspace{\stretch{1}}
