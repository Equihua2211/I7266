%%%%%%%%%%%%%%%%%%%%%%%%%%%%%%%%%%%%%%%%%
% Plantilla para reporte en LaTeX
% Por Daniel Martínez
% dag.mtz.97@gmail.com
%
%
% Basado en:
%
% WikiBooks (http://en.wikibooks.org/wiki/LaTeX/Title_Creation)
%
% License:
% CC BY-NC-SA 3.0 (http://creativecommons.org/licenses/by-nc-sa/3.0/)
%%%%%%%%%%%%%%%%%%%%%%%%%%%%%%%%%%%%%%%%%

% Declarar el tipo de documento
\documentclass[10pt,letterpaper]{exam} 

% Paquetes para idioma, carácteres y símbolos
\usepackage[condensed,math]{iwona}
\usepackage[T1]{fontenc}
\usepackage[spanish, mexico]{babel}
\usepackage{amsfonts}
\usepackage{amssymb}

% Tipo de letra 
% \usepackage{tgbonum}
% \usepackage{tgadventor}

% Márgenes, figuras, gráficos, tablas, listas
%\usepackage{geometry}
\usepackage[top=0.6in, bottom=0.6in, left=0.6in, right=0.6in]{geometry}
\usepackage[justification=centering]{caption}
\usepackage{subcaption}
\usepackage{graphicx}
\usepackage{epstopdf}
\usepackage{tabularx}
\usepackage{float}
\usepackage{enumitem}
%\epstopdfsetup{update} % only regenerate pdf files when eps file is newer
% \usepackage{multirow}
% \usepackage{textcomp}

\usepackage{tikz}

%\usepackage{mathtools}
%\usepackage{siunitx}
%\usepackage{mathrsfs}

% Otros

% \usepackage{hyperref}
% \usepackage[os=win]{menukeys}
% \usepackage{biblatex}
% \addbibresource{sample.bib}

\usepackage{listings}
% \usepackage{fancyhdr}
% \usepackage{listingsutf8}
\usepackage{courier}
% \usepackage{color} %red, green, blue, yellow, cyan, magenta, black, white
\usepackage{xcolor}


% Configuraciones 
% \input{arduinoLanguage.tex}

\definecolor{mGreen}{rgb}{0,0.6,0}
\definecolor{mGray}{rgb}{0.5,0.5,0.5}
\definecolor{mPurple}{rgb}{0.58,0,0.82}
\definecolor{backgroundColour}{rgb}{0.95,0.95,0.92}

\lstdefinestyle{CStyle}{
    backgroundcolor=\color{backgroundColour},   
    commentstyle=\color{mGreen},
    keywordstyle=\color{magenta},
    numberstyle=\tiny\color{mGray},
    stringstyle=\color{mPurple},
    basicstyle=\ttfamily\footnotesize,
    breakatwhitespace=false,         
    breaklines=true,                 
    captionpos=b,        
    frame = single,
    keepspaces=true,     
    %linewidth=6cm,
    numbers=left,                    
    numbersep=10pt,                  
    showspaces=false,                
    showstringspaces=false,
    showtabs=false,                  
    tabsize=2,
    language=C,
    xleftmargin=1cm,
    xrightmargin=12cm
}


% \pagestyle{fancy}

% \renewcommand\lstlistingname{Bloque de código}
% \renewcommand\lstlistlistingname{Índice de códigos}

% \setlength{\headheight}{16pt}

% \rhead{\leftmark}
% \lhead{\rightmark}
% \cfoot{\thepage}
% \renewcommand{\headrulewidth}{0.4pt}
% \renewcommand{\footrulewidth}{0.4pt}

\newcommand{\HRule}{\rule{\linewidth}{0.5mm}} % Comando para líneas horizontales

% \definecolor{mygreen}{RGB}{28,172,0} % color values Red, Green, Blue
% \definecolor{mylilas}{RGB}{170,55,241}

\pointpoints{punto}{puntos}

%%%%%%%%%%%%%%%%%%%%%%%%%%%%%%%%%%%%%%%%%
% Inicio del documento
%%%%%%%%%%%%%%%%%%%%%%%%%%%%%%%%%%%%%%%%%

\begin{document}
\begin{center}
\textsc{\Large Universidad de Guadalajara}\\
\textsc{Centro Universitario de Ciencias Exactas e Ingenierías}\\
\vspace{6mm}
\textbf{Examen extraordinario\\ Programación de Sistemas Embebidos - I7266}\\
\vspace{6mm}
{Nombre: \rule{0.7\textwidth}{0.4pt}} Fecha: \rule{0.15\textwidth}{0.4pt}\\
\vspace{6mm}
\fbox{\fbox{\parbox{6in}{\centering\small \textbf{Indicaciones}\\Las preguntas con más de una respuesta correcta solo contarán como correctas si tienen seleccionadas \textit{todas} las respuestas correctas. El puntaje que otorga cada pregunta viene señalado al inicio de las mismas. El examen se evalúa sobre 100 puntos.}}}
\end{center}
\begin{questions}
    \vspace{5mm}
    \question[10] ¿Qué significan las siglas \textit{USART}?
    \vspace{\stretch{1}}
    \question[10] ¿Cuál es la principal diferencia entre la arquitectura Von Neumann y Harvard?
    \vspace{\stretch{1}}
    \question[10] ¿Qué significa \textit{SoC}?
    \vspace{\stretch{1}}
    \question[10] ¿Cuál es el principal uso del protocolo de comunicaciones \textit{JTAG}?
    \vspace{\stretch{1}}
    \question[10] ¿Qué opción define mejor y de manera general qué es un registro en un microcontrolador? 
    \begin{choices}
        \choice Un espacio determinado en el la dirección inicial de la memoria RAM.
        \choice Un segmento de 8 bits en la ROM.
        \choice Un tipo de memoria de rápido acceso.
        \choice Una dirección de memoria del tamaño de una \textit{palabra}.
    \end{choices}
    \vspace{\stretch{1}}
    \question[10] ¿Cuál es la principal diferencia entre la memoria RAM y ROM? 
    \begin{choices}
        \choice La RAM es no volátil y la ROM es volátil.
        \choice La RAM permite guardar más información que la ROM.
        \choice La ROM es no volátil y la RAM es volátil.
        \choice La ROM permite guardar más información que la RAM.
    \end{choices}
    \vspace{\stretch{1}}
    \clearpage
    \question[10] ¿Qué usa un microcontrolador saber el estado después de una operación en la ALU?
    \begin{choices}
        \choice Un registro.
        \choice Una variable.
        \choice El program counter.
        \choice Una señal interna. 
        \choice El stack.
    \end{choices}
    \vspace{\stretch{1}}
    \question[10] ¿Cuántos bytes caben en una variable tipo \texttt{int64\textunderscore t}? 
    \begin{choices}
        \choice 64
        \choice 32
        \choice 16
        \choice 8
        \choice 4
    \end{choices}
    \vspace{\stretch{1}}
    \question[10] ¿Qué es el lenguaje ensamblador?
    \begin{checkboxes}
        \choice Un lenguaje de bajo nivel en el que se puede programar un microcontrolador.
        \choice El lenguaje en el que se encuentra un archivo \textit{.hex}.
        \choice Un lenguaje que permite usar mnemónicos en lugar de instrucciones de lenguaje máquina.
        \choice Un lenguaje que usa \texttt{1}s y \texttt{0}s para representar instrucciones.
    \end{checkboxes}
    \vspace{\stretch{1}}
    \question[10] ¿Qué hace una interrupción? 
    \begin{checkboxes}
        \choice Depende de si es externa o interna.
        \choice Interrumpe la ejecución del código de manera indefinida.
        \choice Cambia el flujo de ejecución del código para antender un evento.
        \choice Cambia el estado del microcontrolador según una señal externa.
    \end{checkboxes}
    \vspace{\stretch{1}}
\end{questions}
\thispagestyle{empty}
\end{document}

%\vspace{1cm}
%\question ¿Qué hace un compilador? (5 puntos)
%\begin{choices}
%   \choice alsdkfa
%   \choice asdklfjd
%\end{choices}
%\vspace{1cm}
%\question ¿Qué es un adsfasdf? \\
%\begin{oneparchoices}
%    \choice alsdkfa
%    \choice asdklfjd
%\end{oneparchoices}
%\vspace{\stretch{1}}
%\question ¿Qué es un adsfasdf? \\
%\begin{checkboxes}
%    \choice alsdkfa
%    \choice asdklfjd
%\end{checkboxes}
%\question ¿Qué entiendes por el término \textit{arquitectura} en el contexto de microcontroladores? (10 puntos)
%\vspace{\stretch{1}}
%\question ¿Qué es una interrupción y por qué puede ser conveniente usar una en lugar de hacer \textit{polling} a un evento? (10 puntos) 
%\vspace{\stretch{1}}
%\question ¿Cuáles son las diferencias entre una interrupción interna y una externa? (10 puntos)
%\vspace{\stretch{1}}
%\question ¿Qué pasos se requieren para pasar de código fuente (\textit{.c}) a un ejecutable para el ATmega328P (\textit{.hex})? (20 puntos)
%\vspace{\stretch{1}}
%%\clearpage
%\question Explica la diferencia entre memoria \textit{volátil} y \textit{no-volátil}. (5 puntos)
%\vspace{1cm}
%\question ¿Qué significan RAM y ROM? (5 puntos)
%\vspace{1cm}
%\question ¿Las RAM y ROM son memorias \textit{volátiles} o \textit{no-volátiles}? (5 puntos)
%\vspace{1cm}
%\question Explica cuál crees que sea el propósito de las siguientes líneas de código en C para un microcontrolador de la familia AVR de 8 bits. (20 puntos) 
%\begin{lstlisting}[style=CStyle]
%DDRA = 0x0F;
%PORTA = 0xFF;
%\end{lstlisting}
%\vspace{\stretch{1}}
%\question ¿Cuál es el principal uso de la RAM y la ROM en un sistema computacional, respectivamente? (10 puntos)
%\vspace{\stretch{1}}
%\question Si el rango de memoria direccionable de un microcontrolador está delimitado por el ancho del bus de datos, ¿cuál será la última dirección a la que un microcontrolador de 16 bits podría accesar? (20 puntos)
%\vspace{\stretch{1}}
%\question Hay una estructura inicializada en la dirección \texttt{0x007FD8} de la memoria SRAM de un microcontrolador. La estructura está compuesta por una variable tipo \texttt{uint8\textunderscore t}, un arreglo de 8 elementos tipo \texttt{int32\textunderscore t} y una variable tipo \texttt{char}. Si todos los elementos de la estructura se almacenan uno después del otro y sin espacios. ¿Cuál sería la dirección de la variable tipo \texttt{char}? (25 puntos)
%\vspace{\stretch{1}}

